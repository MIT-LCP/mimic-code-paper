\documentclass{elsart}

\usepackage{graphicx}
\begin{document}

\begin{frontmatter}
%\runtitle{Insert a suggested running title}
\title{MIMIC code: a repository for deriving\\ clinical concepts}

\author[Cambridge]{Alistair E. W. Johnson}\ead{aewj@mit.edu},
\author[Cambridge]{Tom J. Pollard}\ead{tpollard@mit.edu}
\address[Cambridge]{Massachusetts institute of Technology, Cambridge}

          
\begin{keyword}
critical care; open data; data mining; secondary use of electronic health records.
\end{keyword}

% Review/rewrite abstract. Focus on importance of the code for analysis.

\begin{abstract}
Secondary analysis of routinely collected data - contrasted with the primary analysis conducted in the process of caring for the individual patient - offers an opportunity to extract knowledge that will lead us towards the goal of optimal care. Retrospective studies frequently require similar clinical concepts, so there is benefit in providing open, standardized tools for deriving these concepts to ensure the consistency and efficiency of future studies. We present the MIMIC Code Repository, a collection of open source code for deriving clinical concepts using the MIMIC Critical Care Database. Concepts include severity of illness scores, organ failure indices, and duration of treatments such as ventilation and dialysis. The MIMIC Code Repository is in active development on GitHub. All code is made available under a permissive MIT license unless otherwise indicated.\\
\end{abstract}

\end{frontmatter}

% Introduction; Results; Discussion; Materials and methods; Acknowledgements; References; Figures with the corresponding legend below each one; and Tables.

\section{Introduction}

% TOPICS TO COVER IN INTRODUCTION:
% - FOCUS ON THE CODE. NO MATTER HOW GOOD THE DATA, CODE IS IMPORTANT FOR GETTING SENSIBLE RESULTS.
% - What is the ICU?
% - Why is data in the ICU an important commodity?
% - What is MIMIC
% - Benefits of creating a centralizing code repository
% - What this paper will discuss: results = concepts currently covered, discussion = collaborative goal of the project

Intensive care units care for the sickest patients in a hospital. Often require support for multiple organs. High mortality rate. High rate of sepsis. Potential for improving outcomes...

There is substantial heterogeneity in intensive care populations, particularly in aspects such as patient physiology, presence of disease, and intervention types. This heterogeneity presents a significant challenge to understanding the relationships between provision of care and patient outcomes, and as a result the impact of many widely-practiced treatments and interventions remains unknown [REFS: Ioannadis]. It has been estimated that [X in X treatments are not supported by evidence]. 

Vast quantities of data are routinely collected by modern hospital monitoring systems and even more so in intensive care units where patients frequently suffer organ failure and require close observation. There is optimism that increasing availability of large scale clinical databases will offer opportunities to overcome many of the challenges associated with heterogeneity and offer new insights into critical care medicine [REF: BIG DATA etc]. 

But in addition to quality data, robust code must be developed to enable analysis. Challenges in interpreting data include deriving severity of illness scores (often used in adjustment), deriving duration of treatments based on [ambiguous?] data, and selecting cohorts based on clinical guidelines. The code used to conduct analysis is critically important in the results.

Typically concepts are coded during analysis, often independently by research groups, often without complete understanding of the data and unable to interact with hospital staff. As a result, the quality of code varies widely. [For example, of X papers using MIMIC in 2013, X different approaches to identifying sepsis were taken.] While controversial, this point was highlighted by NEJM. Numerous examples where errors in code have resulted in retracted papers...

We have implemented an open source library. Novel because... Able to work directly with clinicians to develop the code. Bridge gap between data scientists and clinical environment. Support interpretation of data. Ref NEJM paper highlighting importance of working with clinicians. Also bridging the gap paper. Novel because entire workflow can be reproduced, from data to publication. 

While the case for open data is well publicised, we believe open code is equally important.

% TP: the code that is used in the interpretation and analysis of the data is arguably as important as the data itself. different approaches in coding X can lead to different interpretation of the data. it is important that code is developed by people who are familiar with the environment where the data was collected etc.

% One such database is the Medical Information Mart for Intensive Care (MIMIC-III), collected from patients admitted to intensive care units in the Beth Israel Deaconess Medical Center, Boston, MA, USA \cite{mimiciii}. The latest version of MIMIC-III, v1.3, houses data spanning 11 years between 2001 and 2012, and is made freely available to researchers upon signing of a data use agreement and proof of a human studies training course. 

% MIMIC-III is an unmatched research resource in the area of critical care informatics that promotes crowdsourcing of knowledge generation and sharing of fully reproducible studies.

\section{Results}

A prerequisite for using much of the code in the MIMIC Code Repository is access to the MIMIC-III Database, so we provide scripts to enable researchers to build the MIMIC database in a variety of database systems including PostgreSQL, MySQL, Oracle, and MonetDB. The set of core clinical concepts which have been extracted using structured query language (SQL)\footnote{All queries have been developed and tested using PostgreSQL 9.5.1.} are as follows. Additionally, we seek to provide introduction to the data. Tutorials are provided to give new users and introduction to the MIMIC database. A cookbook of sample code is provided to introduce new users to the database in a friendly way. 

\subsection{Severity of illness scores}

% Explain importance.
% Explain challenges in calculating scores.
% Explain code.
% Include figure showing...

% \emph{Organ failure scores}: Multi-organ failure is a hallmark of acute illness and quantify the morbidity for a given patient. The Sequential Organ Failure Assessment (SOFA) score [REF] and Logistic Organ Dysfunction System (LODS) [REF] both assess six organ systems for failure. Single organ failure scores implemented include MELD [REF], commonly used to determine suitability for a liver transplant, RIFLE [REF], which quantifies acute kidney injury, and KDIGO [REF], also used for acute kidney injury.

Several scoring systems have been developed in order to quantify extent of illness in hospital patients [REF]. These scoring systems are used widely in secondary analysis of health data for a variety of purposes, including selection of study populations and as covariates for severity adjustment in physiological models. While severity scores are integral to many research studies, their definition can present several challenges and it is crucial to recognise the limitations of how scores are generated. Firstly, for example, most severity scores are developed with well curated datasets, usually acquired either through prospective data collection by trained personnel or through manual data abstraction by qualified professionals. As a result, the data tends to be cleaner and often has, perhaps more importantly, a distribution that is markedly different from routinely collected data such as that present in an electronic health record. 

Secondly, routinely collected data often lacks some of data elements required to compute the score. For example, the comorbidity ``''biopsy proven cirrhosis``'' is not simple to determine as there is no routine documentation of this concept in the clinical workflow. Finally, the data definitions for the same concept can vary between the original dataset used to define the severity score and the electronic health records being analyzed. For example, the Glasgow Coma Scale (GCS), a common marker of neurological dysfunction which ranges from 3 (worst) to 15 (best), is usually assumed to be 15 for patients who are unable to be assessed due to sedation or ventilation. In an electronic health record however, this definition is not strictly adhered to as there is no defined protocol, and as a result sedated patients may be assigned a score of 15 by some care providers, and a score of 3 by others.

Working directly with caregivers has helped us to address these issues in the code, helping to ensure the derived severity scores accurately reflect the true severity of illness in patients. There are five severity of illness scores currently implemented in the MIMIC Code Repository: APS-III \cite{aps}, SAPS \cite{saps}, SAPS-II \cite{sapsii}, SOFA \cite{sofa} and OASIS \ref{oasis}. A more detailed comparison of the severity scores is provided in the supplementary material, along with discussion of the assumptions that have been made when defining severity scores. An example of the importance of the variance caused by a non-centralized code base is shown in \ref{fig:SevScoresOverTime}, where the performances of two different implementations of the SOFA score in discriminating hospital mortality are shown. Both of these implementations have been used in previous publications.

\subsection{Duration of interventions}

% Explain importance.
% Explain code.
% Include figure showing data points vs derived duration.

% \emph{Treatment durations}: Studying the effect of treatments on patient health is of great interest, though deriving the timing of these treatments from a database can be non-trivial. We provide views with (i) vasopressor use start and stop times for all vasopressors and individual medications and (ii) mechanical ventilation start and stop times. The duration of these interventions is also a useful measure of treatment intensity.

When carrying out a study it may be necessary or desirable to know the durations of interventions and physiologic states such as mechanical ventilation, vasopressor administration, and hypotension. Deriving these durations from routinely collected data is often non-trivial, requiring a strong understanding of the underlying data as well as the environment in which it was collected. Cao et al highlight this issue in a 2010 paper on heuristics to determine ventilation times of ICU patients, noting that while ``''On the surface, it appears trivial to determine ventilation times``'', ``''when facing the reality of retrospective data, it is not straightforward``'' [REF Cao et al 2010]. In collaboration with staff at the Beth Israel Deaconess Medical Center - the primary source of data in the MIMIC-III database - we have provided code to derive start and stop times for continuous events such as mechanical ventilation, dialysis, and various vasopressors. 

Taking invasive mechanical ventilation as an example, intubation and extubation times are not well documented, so they are determined through the use of surrogate settings which are synchronized with the patient's chart via the ventilator. These settings include flow rates, tidal volumes, and so on. The key assumptions made when defining ventilation are: start time of ventilation is the first occurrence of a ventilator setting, and end time of ventilation is the time of a ventilator setting, followed by at least 8 hours of no ventilator settings. Consequently, short extubations (<8 hours) are not captured by the ventilation query, and the end time of ventilation only approximates the actual extubation time. An example of these durations is provided in Figure \ref{fig:treatment}.

\subsection{Clinical guidelines and definitions of disease}

% Explain importance.
% Explain code.
% Include figure, perhaps cluster.

% \emph{The Angus criteria for defining sepsis}: Sepsis is a serious illness caused by infection and is a major focus of clinical research. Angus criteria utilize billing codes to classify a hospital admission as being related to sepsis [REF TO ANGUS], and the criteria have been recently validated [RECENT ANGUS VALIDATION].

% In addition, scores used to measure the failure of a specific organ are also available for the hepatic (MELD \cite{meld}) and renal (KDIGO \cite{kdigo}, RIFLE \cite{rifle}) systems. RIFLE (Risk, Injury, Failure, Loss of kidney function, and End-stage kidney disease) classification

Numerous clinical guidelines have been developed by well-recognised experts and organisations to assist in the identification and management of specific clinical conditions. These guidelines are often used in clinical studies for risk adjustment and for selection of patient cohorts. One clinical definition implemented in the MIMIC Code Repository, for example, is Angus criteria. These criteria are a widely used definition of severe sepsis, a high-risk complication of infection that consumes considerable healthcare resources and is strongly associated with patient mortality, with a 2001 paper attributing ~215,000 deaths from severe sepsis in the US annually [REF - Angus 2001]. The Angus criteria are based upon hospital billing codes, making them relatively simple to implement, but other guidelines are less straightforward.

The Glasgow Coma Scale (GCS), for example, represents the level of conciousness of a patient, and as such it is influenced by level of sedation. In the collection of data for severity scoring, values of GCS were set to 15 (normal) if the care provider felt the GCS was not a true reflection of the patient's neurological status. This situation would occur if the patient was sedated or if a tracheostomy prevented a verbal response. However, these values in MIMIC-III are often recorded as 3 (extremely abnormal) - in particular, the string for verbal response can be either ``''1.0 No response``'' or ``''1.0 ET/Trachy``''. Simply extracting the GCS as it appears in MIMIC would be incongruent with the original definition of the scale and would likely compromise their discrimination and calibration. Ideally, all patients who have low GCS due to sedation would have their value replaced by 15, but in practice determining sedation status from a patient's chart is a difficult task.

In addition to implementing code for the Angus criteria for severe sepsis and Glasgow Coma Scale for neurological status, we provide scripts for a growing number of additional clinical guidelines. These include the Kidney Disease: Improving Global Outcomes (KDIGO) classification for acute kidney injury, a common, harmful, and potentially treatable condition characterised by abrupt decrease in kidney function [REF - KDIGO guidelines], as well as the Model For End-Stage Liver Disease (MELD) Score, which is commonly used in the care of patients with cirrhosis for assessing the severity of chronic liver disease. Critically ill patients frequently have many comorbidities which influence both their overall health and their trajectory of health during an individual hospital stay. To support analysis that seeks to capture the variation in patient comorbidities, the MIMIC Code Repository includes code for computing the Elixhauser Comorbidity Index, a clinical definition that seeks to summarize the level of comorbidities in individual patients using billing codes collected at hospital discharge [REFS].

% \begin{figure}[!tpb]%figure1
% \centerline{\includegraphics[width=0.5\textwidth]{figures/SOFA.eps}}
% \caption{Discrimination of two implementations of SOFA across fiscal years as measured by the area under the receiver operator characteristic curve (AUROC).}\label{fig:SevScoresOverTime}
% \end{figure}

\section{Discussion}

% Discuss use of code repository and data in university courses.



% While the case for open data is well publicised, open code is equally important. http://www.nature.com/nature/journal/v482/n7386/full/nature10836.html
% Code is as important as the data. Provide some examples.
% Should be developed in collaboration with people who understand the data. Discuss NEJM research parasite paper. The first concern is that someone not involved in the generation and collection of the data may not understand the choices made in defining the parameters. Symbiotic relationship. Closes gap from clinical practice to data analysis. Frequent discussion with hospital staff.

% We have presented a suite of code simplifying data extraction and facilitating clinical studies. The repository becomes a high-quality resource for researchers in the field. Already used by researchers. X contributors and X forks.



% - Open source nature improves validity/allows for community cross-checking
% - Tell 'em about your dream Martin!


%[Briefly say here that in the past people have not shared their code systematically, which is bad. Leads on to next section.]

% Analysis of critical care data often requires definition of clinical concepts, such as severity of illness scores, organ failure indices, and duration of treatments including ventilation and dialysis.Historically, this code has been produced by independent researchers; a process which is time consuming, inefficient, and error-prone.

% Furthermore, many decisions are made when extracting clinical concepts, and these decisions may have a large impact on the resultant analysis. One example is code written to extract the Glasgow Coma Scale (GCS). Severity of illness scores stipulate a value of 15 (``normal'') should be assigned when the GCS cannot be obtained (e.g. due to patient sedation). However, it is common for clinical staff to record values of 3 for the GCS of sedated patients. This results in a systematic bias for sedated patients unless appropriate measures are taken to correct the recorded values. Even if correctly performed, these details would likely be omitted from publications, making study methods difficult to reproduce.

%Typically the code used to generate these concepts is generated by the researcher and often not widely shared for reuse or validation. The result is that work is not easily reviewed and reused, leading to inconsistent approaches between projects, repeated work, and increased probability of errors in the extraction process. 

% There is a great opportunity for unifying studies on MIMIC with the creation of a centralized repository for data extraction code. Here we describe such a centralized repository, with the hope that is fulfills this need and enhances the reproducibility of research on the MIMIC-III database.

% Alistair: this is nice to say but feels a little redundant
%The MIMIC Code Repository helps to address these issues by providing a centralised and collaborative codebase for the research community. By allowing researchers to collaboratively develop, share, and review code, the repository helps to standardize the definition of key clinical concepts, promotes continuous review and improvement, and facilitates future studies on the MIMIC-III database. 


% Analysis of the data collected by these systems has the potential to increase medical knowledge, taking us towards improved care and patient outcomes.

% help to provide a solution to... Analysis often requires matching patients based on severity, selecting cohorts based on presence of comorbidities, etc. There is therefore benefit in developing reusable code for... Quantifying the severity of illness for a patient is an integral part of retrospective analysis as it allows for investigation of concepts of interest in comparable patient populations. 

% Set out some of the problems and our knowledge gaps. Emphasise importance of research. For example, preventable mortality, reducing length of stay, preserving resources etc. Introduce some studies that have used MIMIC. Explain that code has been written independently on each occasion. 

% To address these gaps in our knowledge, key steps are needed. Available data. Studies that can be reviewed and improved and repeated over new datasets. Facilitated by code sharing etc..

% Discuss creating a reproducible workflow for research.

% Emphasise that the code will be reused widely in research studies. Also emphasise how much work has gone into creating the code. What are the challenges of creating the scores?

% Numerous research studies have been carried out on the MIMIC database in the past, but the code used in analysis has largely been developed independently and often not shared. Lack of code sharing is typical of scientific software development and is a widely recognized issue [Ref: http://www.nature.com/news/2010/101013/full/467775a.html]. By creating the MIMIC Code Repository we have provided a central hub for development of clinical concepts, which we believe will help to standardize and improve future analyses.
% The repository is intended to be a continuously developed resource enhancing the sustainability of the code and creating a community around data analysis in MIMIC.

% The code described herein follows guidelines for good practice in scientific programming, including incremental development with a distributed version control system, unit tests, and a public issue tracker [Ref: G Wilson paper].

%It is typical for PhD or Postdoctoral researchers to write code for a project without thought of long term usage, often leading to unmaintained and unusable code. The MIMIC Code Repository is intended to be a continuously developed resource. We anticipate its importance to grow over time, as the content increases. 

%Openly available code is a key step in improving the quality of research as it provides validity to analyses performed and allows progressive improvement of code. While rare in clinical research, the combination of open data and code is powerful. Studies can be made reproducible and code is available to be reviewed and developed. Importantly, open resources such as MIMIC become important learning resources too: there are now multiple university courses around the world in which MIMIC forms a core learning tool. 



%\subsection{Collaboration}

% Conducting effective clinical research calls for a broad range of knowledge, covering areas such as clinical practice, data collection processes, and analytical methods. For this reason, we believe that it is important to nurture collaboration between disciplines including caregivers, computer scientists, and data analysts.

%A key challenge in developing the code within the repository is in understanding the nuances of the underlying hospital data, so the development process has involved close collaboration between hospital staff and data scientists. This cross disciplinary partnership is facilitated when code is shared openly and developed as a community. Mention how StackExchange is being used for discussion. Has this happened for other research projects? I guess so, but would be interesting to check.

%\subsection{Quality and sustainability}

%unit tests check the operation of software upon addition of new code to the repository, and a public issue tracker allows research related questions to be raised, encouraging community development and helping to ensure that the code is sustainable [Ref: G Wilson paper]. Often use Jupyter Notebook to create database connection to carry out studies. Continuous review. Services like Github provide incentives for researchers to contribute.

% The MIMIC Code Repository is an important resource for researchers working with the MIMIC critical care database. The repository provides code for deriving a variety of clinical concepts and will continue to incorporate new concepts as they are calculated, allowing for rapid prototyping of clinical questions in a large retrospective database. 
% Fully reproducible analytical workflow are encouraged by the openly available nature of both the data and code.
% Finally, the code written is modular and generalisable, and may support research in other similarly structured clinical databases.



\section{Materials and methods}

% - We follow good software practices etc
% - Github. cite github paper.
% - Modular approach to generating stuff
% - Describe the structure of the repository
% - Issue tracking

% Detail on the creation of the severity of illness scores
% Angus criteria
% DRG codes


\begin{ack}
The authors would like to thank Professor Roger G. Mark, the MIT Laboratory for Computational Physiology, Philips Healthcare and the Beth Israel Deaconess Medical Center for the creation of the MIMIC-III database.%\vspace*{-12pt}
\end{ack}

\section*{Funding}

This work has been supported by grants NIH-R01-EB017205, NIH-R01-EB001659, and NIH-R01-GW104987 from the National Institutes of Health.%\vspace*{-12pt}

\bibliographystyle{plain}
\bibliography{refs}

%\appendix
%\section{Some stuff}
\end{document}




\begin{figure}
\begin{center}
%\includegraphics[height=4cm]{jcaesar.eps}    % The printed column  
%\caption{Gaius Julius Caesar, 100--44 B.C.}  % width is 8.4 cm.
\label{fig1}                                 % Size the figures 
\end{center}                                 % accordingly.
\end{figure}

% OR

%\begin{figure}
%\begin{center}
%\epsfig{file=jcaesar,width=7cm}
%\caption{Gaius Julius Caesar, 100--44 B.C.}
%\label{fig1}
%\end{center}
%\end{figure}
